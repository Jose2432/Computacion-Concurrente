\documentclass[answers]{exam}

% Símbolos
\usepackage{recycle}
\usepackage{amsmath}
\usepackage{amsfonts}
\usepackage{amssymb}

% definimos el diccionario del idioma
\usepackage[spanish]{babel}

% Figuras
\usepackage{graphicx}

% Márgenes
\addtolength{\voffset}{-1cm}
\addtolength{\hoffset}{-1.5cm}
\addtolength{\textwidth}{3cm}
\addtolength{\textheight}{2cm}

%% Definimos lema.
\newtheorem{lemma}{Lema}

% Información del Encabezado
\runningheader{Computación Concurrente}{Práctica 03}{\today}
\runningheadrule{}
\footer{}{Página \thepage\ de \numpages}{}
\footrule{}
\thispagestyle{headandfoot}
\renewcommand{\solutiontitle}{\noindent\textbf{Solución:}\enspace}

% Punto decimal
\decimalpoint

\begin{document}

\input{portada.tex}
\newpage


\section{Teoría:}
\begin{enumerate}
    \item Proponer 4 problemas donde se pueda utilizar el algoritmo de Peterson para su solución.
    \begin{solution}
        \begin{itemize}
            \item Gestionar el uso de una computadora para dos roomies.
            \item Controlar las horas de salida de dos personas.
            \item Administrar el uso de un hangar para dos avionetas en un pequeño aeropuerto.
            \item Dar prioridad a uno de los dos procesos para un recurso que gestiona las llegadas de entrada en una empresa.
        \end{itemize}
    \end{solution}

    \item Proponer 2 problemas donde se pueda utilizar el Algoritmo del Filtro
    \begin{solution}
        \begin{itemize}
            \item Dar prioridad al préstamo de libros a usuarios con suscripción en una librería.
            \item Priorizar $n$ procesos sobre algunos recursos en un sistema operativo.
        \end{itemize}
    \end{solution}

    Responde las siguientes preguntas, justificando tu respuesta:
    \begin{itemize}
        \item ¿Los algoritmos cumplen con No Deadlock?
        \begin{solution}
            Sí cumplen porque los procesos primero ingresarán a la sección crítica con seguridad, por lo tanto, si un proceso se adelanta entonces éste ingresaría a la sección crítica.
        \end{solution}

        \item ¿El Algoritmo de Peterson cumple con la propiedad de Justicia?
        \begin{solution}
           Si cumple la propiedad de justicia, ya que en el algoritmo de Peterson cuando un proceso  y quieren entrar  a su región crítica este mediante las banderas dirán que quiere entrar a su región  para después ceden su  turno y se quedará  en una espera ocupada donde estará preguntando si es el turno del otro proceso y si el otro proceso quiere usar la región crítica  y esto sucederá  hasta que el otro proceso deje de usar la región crítica donde después de ocupar la región crítica se dirá mediante la bandera que no se quiere entrar a la región crítica por lo que el que el proceso que estaba en la espera ocupada  saldrá de la espera ocupada para así entrar a su región crítica por lo que  en algún momento cuando el proceso que esté en su región crítica salga  entonces el otro podrá entrar a su región crítica y si un proceso está en su región no crítica esto no afectará al que quiera acceder a su región crítica ya que mediante la  bandera del proceso que está en su región esta indicara que no no quiere entrar a su región crítica por lo que el que quiere entrar a su región crítica saldra de la espera ocupada para entrar a su región crítica.
        \end{solution}

        \item ¿Cuál de estos algoritmos cumple con la propiedad Libre de Hambruna?
        \begin{solution}
           \begin{itemize}
               \item  El algoritmo de Peterson
               \item El algoritmo del Filtro
           \end{itemize}
        \end{solution}

        \item ¿Cumplen con Exclusión Mutua?
        \begin{solution}
            Sí cumplen ya que se encargan de gestionar los procesos para que usen los recursos de manera organizada y con la finalidad que no se produzcan problemas.
        \end{solution}

    \end{itemize}
\end{enumerate}

\section{Referencias:}
\begin{itemize}
    \item OS Paterson Solution - javatpoint. (s.f.). www.javatpoint.com. Recuperado 23 de octubre de 2022,\\ de https://www.javatpoint.com/os-paterson-solution
    \item (S/f). Utexas.edu. Recuperado el 23 de octubre de 2022, de\\ http://users.ece.utexas.edu/~garg/dist/jbkv2/Peterson-Proof.pdf
    \item Exclusión mutua - Win32 apps. (2022, 24 septiembre). Microsoft Learn. Recuperado 23 de octubre de 2022, de\\ https://learn.microsoft.com/es-es/windows/win32/wmformat/mutual-exclusion
     \item seguridad y vivacidad/viveza - PDF Free Download. (s. f.). Recuperado 29 de octubre de 2022, de\\ https://docplayer.es/203743104-Seguridad-y-vivacidad-viveza.html
    \item Algoritmo de Peterson - frwiki.wiki. (s. f.). Recuperado 29 de octubre de 2022, de\\ https://es.frwiki.wiki/wiki/Algorithme_de_Peterson
\end{itemize}


\end{document}
